\documentclass[ngerman, parskip=half]{scrartcl}
\usepackage{fontspec}
%\setmainfont{Latin Modern Roman}
\usepackage{lmodern}
\usepackage{babel}
\usepackage{mathtools, amssymb, dsfont, amsthm}
\usepackage[mathcal]{eucal}
\usepackage{tikz-cd}
\usepackage{enumitem}
\newcommand*{\embrace}[1]{(#1)}
\usepackage{scrlayer-scrpage}
\usepackage[backend=biber]{biblatex}
\bibliography{spanier_whitehead_duality.bib}
\usepackage{hyperref}
\hypersetup{%
  pdfauthor=Gesina Schwalbe,
  pdftitle=Spanier-Whitehead Duality}

\theoremstyle{definition}
\newtheorem*{Def}{Definition}
\newtheorem*{Bsp}{Beispiel}
\newtheorem*{Thm}{Theorem}
\newtheorem*{Kor}{Korollar}
\newtheorem*{Lem}{Lemma}


\newcommand*{\SP}{\ensuremath{\mathcal{SP}}}
\newcommand*{\SPf}{\ensuremath{\SP_{\mathrm{F}}}}
\newcommand*{\Smash}{\ensuremath{\wedge}}
\newcommand*{\Wedge}{\ensuremath{\vee}}
\newcommand*{\SDual}{S-Dual}
\newcommand*{\dual}[1]{\ensuremath{#1^*}}
\newcommand*{\Dd}[1]{\ensuremath{D_{#1}}}
\newcommand*{\dD}[1]{\ensuremath{{}_{#1}D}}
\newcommand*{\uD}[1]{\ensuremath{{}^{#1}\!D}}
\newcommand*{\Du}[1]{\ensuremath{D^{#1}}}
\newcommand*{\Sph}{\ensuremath{S^0}}
\newcommand*{\htpic}{\simeq}
\newcommand*{\longto}{\longrightarrow}
\newcommand*{\N}{\mathds{N}}
\newcommand*{\Z}{\mathds{Z}}
\newcommand*{\R}{\mathds{R}}
\newcommand*{\id}{\text{id}}

\newcommand*{\D}[1]{\mathrm{D}(#1)}
\newcommand*{\Th}[1]{\text{Th}(#1)}
\newcommand*{\Sb}[1]{\mathrm{S}(#1)}

\begin{document}
\clearpairofpagestyles
\ohead*{24.05.2017, Universität Regensburg}
\ihead*{Gesina Schwalbe}
\cfoot*{\pagemark}

\vspace*{.5eM}
\begin{center}
  Seminar zur stabilen Homotopietheorie – Vortragsnotizen\par
  {\LARGE\bfseries Spanier-Whitehead Dualität}
\end{center}
\bigskip

\begin{Def}[S-Dualität]
  Für $X\in\SP$ heißt $\dual{X}\in\SP$ zusammen mit einem Morphismus
  $\mu\colon X\Smash \dual{X}\to\Sph$ (bzw. äquivalent
  $\rho\colon\Sph\to X\Smash\dual{X}$) ein \SDual{} von $X$, falls für
  alle $U,V\in\SP$ die Gruppenhomomorphismen $\Dd{\mu},\dD{\mu}$
  (bzw. $\uD{\rho},\Du{\rho}$) Isomorphismen sind:
  \begin{alignat*}{3}
    &\Dd{\mu}\colon &[U, V\Smash\dual{X}] &\longto [U\Smash X,V]
    &:\uD{\rho}\\
    && f &\longmapsto
           \left(
           U\Smash X \xrightarrow{f\Smash 1}
           V\Smash(\dual{X}\Smash X) \xrightarrow{1\Smash\mu}
           V\Smash\Sph \htpic V
           \right) \\
    &\dD{\mu}\colon &[U, V\Smash X] &\longto [U\Smash\dual{X},V]
    &:\Du{\rho}\\
  \end{alignat*}  
\end{Def}

\begin{Lem}[Eindeutigkeit]
  Ein \SDual{} $X'$ zu $X\in\SP$ ist eindeutig bis auf Homotopie,
  geschrieben $\dual{X}$
  (nutze Brownsche Darstellbarkeit von $[-\Smash X,\Sph]$).
\end{Lem}

\begin{Def}[\SDual{} von Abbildungen]
  Für $X,Y\in\SP$ mit \SDual{}en $(\dual{X},\mu),(\dual{Y},\nu)$
  ist das \SDual{} einer Abbildung $f\in[X,Y]$ definiert als
  $\dual{f}\coloneqq\Theta(\mu,\nu)(f)\in[\dual{Y},\dual{X}]$ durch
  den Isomorphismus
  \begin{gather*}
    \SwapAboveDisplaySkip
    \Theta(\mu,\nu)\colon
    [X,Y]
    \overset{\sim}{\underset{\dD{\nu}}{\longrightarrow}}
    [\dual{Y}\Smash X, \Sph]
    \overset{\sim}{\underset{\Dd{\mu}}{\longleftarrow}}
    [\dual{Y},\dual{X}]
  \end{gather*}
\end{Def}

\begin{Lem}
  Das \SDual{} verhält sich gutartig bzgl. $\Sigma$, Doppeldual,
  Komposition und $\Smash$:
  Seien $X,Y,Z\in\SP$, dazu
  $(\dual{X},\mu)$, $(\dual{Y},\nu)$, $(\dual{Z},\eta)$ \SDual{}e
  und $f\in[X,Y]$, $g\in[Y,Z]$. Dann (nachrechnen):
\begin{align*}
  \text{Triviales Dual}&
  &\dual{(\Sigma^{p}\Sph)} &= (\Sigma^{-p}\Sph,
                             \Sigma^{-p}\Sph\Smash\Sigma^{p}\Sph\htpic\Sph)
                             \quad\text{für $p\in\Z$}\\
  \text{\Sigma-Operator}&
  &\dual{(\Sigma^{p}X)} &= (\Sigma^{-p}\dual{X},\mu)
                          \quad\text{und}\quad\dual{\Sigma^p f} =\Sigma^{-p}\dual{f}\\
  \text{Doppeldual}&
  &\dual{(\dual{X})} &= (X,\mu),
                       \quad\text{und}\quad\dual{(\dual{f})} = f,
                       \quad\dual{\id_X}=\id_{\dual{X}}\\
  \text{\Smash-Produkt}&
  &\dual{(X\Smash Y)} &= (\dual{X}\Smash\dual{Y}, \mu\Smash\nu)\\
  \text{Komposition}&
  &\dual{(g\circ f)} &= (\dual{f}\circ\dual{g})
\end{align*}
\end{Lem}

\begin{Lem}[Kofasersequenzen]
  Das \SDual{} endlicher Spektren ist kompatibel mit Kofasersequenzen,
  d.\,h. für $X,Y\in\SPf$ mit endlichen \SDual{}en
  $(\dual{X},\mu),(\dual{Y},\nu)$ und eine Kofasersequenz (d.\,h. $Z\htpic Cf$)
  \begin{alignat*}{4}
    X &\overset{f}{\longto} \,Y&&\overset{g}{\longto} \,Z&&\overset{h}{\longto}&&\,\phantom{(}\Sigma X
    \intertext{existiert ein \SDual{} $(\dual{Z},\eta)$, s.\,d.}
    \dual{X} &\overset{\,\dual{f}}\longleftarrow
    \dual{Y}&&\overset{\,\dual{g}}\longleftarrow
    \dual{Z}&&\overset{\,\dual{h}}\longleftarrow &&\,\dual{(\Sigma X)}=\Sigma^{-1}\dual{X}
  \end{alignat*}
  wieder eine Kofasersequenz ist.
  \begin{proof}[Beweisidee]
    Da $X,Y,\dual{X},\dual{Y}$ endlich sind,
    sind sie Einhängungsspektren von CW-Kom\-ple\-xen mit entsprechenden
    Kofasersequenzen.
    Finde damit einen erzeugenden CW-Kom\-plex für $\dual{Z}$ und eine
    erzeugende Abbildung für $\eta$.
  \end{proof}
\end{Lem}

\begin{Thm}[Existenz]
  Für endliche Spektren existiert das \SDual{} und ist endlich.
  \begin{proof}[Beweisidee]
    Sei $X\in\SPf$. Da $X$ endlich viele Zellen hat, ist
    Induktion über die Anzahl der Zellen möglich:
    Für den Induktionsanfang betrachte ein Spektrum mit nur einer
    Zelle. Dies ist homotop zu $\Sigma^{p}S^q$ für passende
    $p,q\in\Z$, welches ein endliches \SDual{} besitzt.
    Für den Induktionsschritt nutze, dass das Ankleben einer Zelle
    eine Kofasersequenz ist und man das vorige Lemma anwenden kann.
  \end{proof}
\end{Thm}

\begin{Kor}[Dualität von Ko-/Homologie]
  Für $E, X\in\SP$ mit \SDual{} $(\dual{X},\mu)$ von $X$ erhalte
  Isomorphismen
  \begin{alignat*}{2}
    \Dd{\mu}\colon&&
    [\Sigma^{p}\Sph,E\Smash\dual{X}] = E_{p}(\dual{X})
    &\overset{\sim}{\longrightarrow}
      E^{-p}(X) = [\Sigma^{p}\Sph \Smash X, E]\\
    \dD{\mu}\colon&&
    E_{p}(X)
    &\overset{\sim}{\longrightarrow}
      E^{-p}(\dual{X})
  \end{alignat*}
\end{Kor}

\begin{Bsp}
  Sei $M^n$ eine geschlossene, glatte Mannigfaltigkeit mit Einbettung
  $M\subset\R^{n+k}$ für ein $k\in\N$, zugehörigem Normalenbündel
  $\nu^k$ und Thom-Raum $\Th{\nu}$.
  $\nu$ definiert eine Schlauchumgebung
  $\D{\nu}\simeq N\subset\R^{n+k}$ von $M$ und eine Kollapsabbildung
  \begin{gather*}
    \pi\colon
    S^{n+k}=(R^{n+k})^* \overset{\text{pr}}\longrightarrow
    S^{n+k}/(S^{n+k}\setminus N^{\circ}) \simeq
    % N/\partial N \simeq
    % \D{\nu}/S{\nu} \eqqcolon
    \Th{\nu}
  \end{gather*}
  Sei $\bar{\Delta}\colon \Th{\nu}\to \Th{\nu}\Smash \D{\nu}$ die
  von der Diagonalen
  \begin{gather*}
   \Delta\colon (\D{\nu},\Sb{\nu})\to (\D{\nu}^2,\Sb{\nu}\times \D{\nu}) 
  \end{gather*}
  induzierte Abbildung. Dann erhalten wir insgesamt
  \begin{gather*}
    \rho_{n+k}\colon
    S^{n+k} \overset{\pi}\longrightarrow
    \Th{\nu} \overset{\bar{\Delta}}\longrightarrow
    \Th{\nu} \Smash \D{\nu}_+ \xrightarrow{1\Smash \text{pr}}
    \Th{\nu} \Smash M_+
  \end{gather*}
  Die induzierte Abbildung auf den induzierten CW-Spektren (nach
  Morsetheorie ist $M$ als geschlossene, glatte Mannigfaltigkeit
  homotop zu einem endlichen CW-Komplex)
  % \begin{align*}
  %   \rho \in \left[\Sigma^{n+k}\Sph, \Sigma^{\infty}(\Th{\nu}\Smash M_+)\right]
  %   &= \left[\Sph, \Sigma^{-n-k}\Sigma^{\infty}\left(\Th{\nu}\Smash M_+\right)\right]\\
  %   &= \left[\Sph, \Sigma^{-n-k}(\Sigma^{\infty}\Th{\nu})\Smash \Sigma^{\infty}(M_+)\right]
  % \end{align*}
  \begin{gather*}
    \rho \in \left[\Sigma^{n+k}\Sph, \Th{\nu}\Smash M_+\right]
    = \left[\Sph, \Sigma^{-n-k}\left(\Th{\nu}\Smash M_+\right)\right]
    = \left[\Sph, \left(\Sigma^{-n-k}\Th{\nu}\right)\Smash M_+\right]
  \end{gather*}
  ist eine \SDual-Abbildung, d.\,h. $\Sigma^{-n-k}\Th{\nu}$
  repräsentiert das \SDual{} von $M_+$ (als Einhängungsspektrum).
\end{Bsp}

\nocite{*}
\printbibliography
\end{document}

