\documentclass[english]{scrartcl}
\usepackage{fontspec}
% \setmainfont{Latin Modern Roman}
\usepackage{lmodern}
\usepackage{babel}
\usepackage{mathtools, amssymb, dsfont, amsthm}
\usepackage[mathcal]{eucal}
\usepackage{tikz-cd}
\usepackage{enumitem}
\setlist[description]{font=\normalfont\itshape}
\newcommand*{\embrace}[1]{(#1)}
\usepackage{scrlayer-scrpage}
\usepackage[backend=biber]{biblatex}
\bibliography{localisation_completion.bib}
\usepackage{hyperref}
\hypersetup{%
  pdfauthor=Gesina Schwalbe,
  pdftitle=Stable Homotopy Theory II - Localisation and Completion of Spectra}

\theoremstyle{definition}
\newtheorem*{Def}{Definition}
\newtheorem*{DefLem}{Definition/Lemma}
\newtheorem*{Rev}{Revision}
\newtheorem*{Thm}{Theorem}
\newtheorem*{Prop}{Proposition}
\newtheorem*{Cor}{Corollary}
\newtheorem*{Lem}{Lemma}
\newtheorem*{Rem}{Remark}
\theoremstyle{remark}
\newtheorem*{Ex}{Example}
\newtheorem*{App}{Application}

% general language
\newcommand*{\idest}{i.\,e.\ }
\newcommand*{\suchthat}{s.\,t.\ }
\newcommand*{\forexample}{e.\,g.\ }
\newcommand*{\ifftext}{iff}
\newcommand*{\newdef}[1]{\emph{#1}}

% general math
\newcommand*{\N}{\mathds{N}}
\newcommand*{\Z}[1][]{\mathds{Z}_{#1}}  % integers and their localisations
\newcommand*{\Zmod}[1][2]{\Z/#1}
\newcommand*{\R}{\mathds{R}}
\newcommand*{\C}{\mathds{C}}
\newcommand*{\Quaternions}{\mathds{H}}
\newcommand*{\Octonions}{\mathds{O}}
\newcommand*{\id}{\text{id}}
\newcommand*{\longto}{\longrightarrow}
\newcommand*{\htpic}{\simeq}
\newcommand*{\smashed}{\wedge}
\newcommand*{\wedged}{\vee}
\newcommand*{\Cone}[1]{C(#1)}

% specific
\newcommand*{\Spectra}{\mathcal{S}} % htpy category of CW-spectra
\newcommand*{\Sph}{S} % sphere (spectrum)
\newcommand*{\susp}[2][1]{\Sph^{#1}#2}  % suspension
\newcommand*{\conem}[2][]{s^{#1}(#2)}  % map to the mapping cone in cofibre seq
\newcommand*{\Ps}{P}  % set of primes
\newcommand*{\PS}{\mathcal{P}} % set of all primes
\newcommand*{\graded}[1]{#1_*}  % with grading (e.g. category)
\newcommand*{\Abgr}{\graded{\text{Ab}}}  % graded abelian groups
\newcommand*{\Hiso}{$H$-isomorphism}  % H-iso
\newcommand*{\pistar}[2][*]{\pi_{#1}(#2)}  % stable htpy grps
\newcommand*{\timesZ}[1][\Ps]{\otimes_{\Z} \Z[#1]}  % tensor product with Z
\newcommand*{\LP}[1][\Ps]{L_{#1}}  % stable localised htpy functor
\newcommand*{\compl}[2][\Ps]{\hat{#2}_{#1}}  % completion
\DeclareMathOperator{\wlim}{wlim}  % weak limit of spectra
\newcommand*{\A}{\mathcal{A}}  % small subcategory of spectra wt f.t.

\begin{document}
\clearpairofpagestyles
\ohead*{20/12/2017, University of Regensburg}
\ihead*{Gesina Schwalbe}
\cfoot*{\pagemark}

\begin{center}
  Seminar on Stable Homotopy Theory II\par
  \medskip {\Large\bfseries Localisation and Completion of Spectra}
\end{center}
\smallskip

This short introduction gives an overview on how several concepts
known from the algebra of groups can be ported to the category of
spectra.

\section{Localisation by a functor}
\begin{Rev}[Localisation of $\Z$-modules]
  \begin{itemize}
  \item For a $\Z$-module $G$, and $S\subset\Z$ a multiplicatively
    closed subset, the following sequence of canonical maps is exact:
    \begin{gather*}
      0 \longto G[S] \longto G \longto S^{-1}G\coloneqq G\otimes
      S^{-1}\Z
    \end{gather*}
    This means localisation eliminates torsion.
  \item Thus, for $S=\Z/\bigcup_{p\in\Ps}(p)$ with
    $\Ps\subset\PS\coloneqq\{\text{primes in $\Z$}\}$,
    $G_P\coloneqq S^{-1}G$ will at most admit $P$-torsion.
  \end{itemize}
\end{Rev}
\begin{Rev}[Graded Categories]
  \begin{itemize}
  \item A category is \newdef{graded} if it has a self-equivalence
    (suspension), \forexample the graded abelian groups $\Abgr$ with
    index shift or $\Spectra$ with suspension. A functor between
    graded categories is called \newdef{graded} or \newdef{stable}, if it
    commutes with the suspensions.
  \item The homotopy category $\Spectra$ of CW-spectra admits for any
    map $f\colon X\to Y$ \newdef{cofibre sequences} (called exact triangles)
    \begin{gather*}
      {\color{gray} \dotsb \longto \susp[-1] \Cone f
        \xrightarrow{\conem[-1] f} } X \overset f \longrightarrow Y
      \xrightarrow{\conem f} \Cone f \xrightarrow{\conem[2]f} \susp X
      {\color{gray} \longto \dotsb } \;.
    \end{gather*}
    A morphism will be a
    \begin{description}[nosep]
    \item[monomorphism] (\idest for $g,h\colon Z\to X$ with
      $f\circ g=f\circ h$ get $g=h$) \ifftext{} $\conem f=0$,
    \item[epimorphism] (\idest for $g,h\colon Z\to X$ with
      $g\circ f=h\circ f$ get $g=h$) \ifftext{} $\conem[-1] f = 0$.
    \end{description}
    A functor to a graded abelian category is called exact if it is
    stable and maps a cofibre sequence to an exact sequence.
  \end{itemize}
\end{Rev}

From here onwards let $H\colon\Spectra\to\Abgr$ be an
\newdef{exact functor}, $X,Y\in\Spectra$, and $f\colon X\to Y$ a
morphism.

\begin{Def}
  \begin{itemize}
  \item $f$ is called an \newdef{\Hiso{}} if $H(f)$ is an isomorphism.
  \item $f$ is called the \newdef{$H$-localisation} of $X$ if it is terminal
    amongst \Hiso s from $X$. This makes
    the $H$-localisation unique up to equivalence in case it exists
    and $X_H\coloneqq Y$.
  \item $W\in\Spectra$ is called \newdef{$H$-acyclic} if $H(W)=0$.
  \item $Y$ is called \newdef{$H$-local} if $[W, Y]=0$ for any
    $H$-acyclic spectrum $W$.
    $\Spectra_H$ is the full subcategory of $H$-local spectra of $\Spectra$.
  \end{itemize}
\end{Def}

\begin{Rem}
  $H$-localisations exist for all $X\in S$ if $H$ is either
  \begin{description}
  \item[homology functor,] \idest covariant, exact, stable, and commutes with
    coproducts, or
  \item[cohomology functor (represented by $W$),] \idest
    contravariant, exact, stable, and sends coproducts to products,
    with the additional property that for all spectra $X$ and
    $\alpha\colon X\to W$, there is $r\in\N$ \suchthat
    $X\xrightarrow{\alpha} W\to W[-\infty, r]$ is non-zero, where
    $W[-\infty,r]$ is the $r$th Postnikov-tower of $W$.
  \end{description}
  \begin{proof}
    \cite[Prop. 7.6, and Thm. 7.7]{margolis}
  \end{proof}
\end{Rem}

Analoguesly to groups we have the following characterisation of being
an $H$-localisation.
\begin{Prop}\label{thm:charhlocal}
  $f$ is an $H$-localisation \ifftext{} $Y$ is local.
  Thus the $H$-localisations are uniquely determined by the $H$-local
  resp. $H$-acyclic spectra.
  \begin{proof}[proof (scetch)]
    \cite[compare][Lemma 7.1]{margolis}
    \begin{description}
    \item[$\Rightarrow$]
      Let $W\in\Spectra$ be $H$-acyclic, $g\colon W\to Y$ morphism. In
      order to show that $g=0$ it suffices to show $\conem g$ is a
      monomorphism. Applying $H$ to the cofibre sequence of $g$ yields
      that $\conem g$ is an \Hiso, thus also $\conem g \circ f$.
      The existence of a retraction for $\conem g$ then follows from
      terminality of $f$.
    \item[$\Leftarrow$]
      Let $g\colon X\to Z$ be an \Hiso. For terminality of $f$ one
      needs to show that $f$ uniquely factors over $g$, which is
      implied if $g^*\colon [Z,Y]\to [X,Y]$ is an isomorphism.
      The latter holds by exactness of the cohomology functor
      $[-,Y]$.
      \qedhere
    \end{description}
  \end{proof}
\end{Prop}
\begin{Cor}
  \begin{enumerate}
  \item The $H$-localisation of $X$ is initial amongst morphisms
    from $X$ to an $H$-local spectrum.
  \item Thus, every map $f\colon X\to Y$ has a unique $H$-localisation
    $f_H\colon X_H\to Y_H$, s.t.
    $f_H\circ l_x = l_H\circ f\colon X\to Y_H$
    if both $H$-localisations $l_X$, $l_Y$ exist.
  \item If all $H$-localisations exist, we get an exact functor
    $\Spectra\to \Spectra_H$, $X\mapsto X_H$, $f\mapsto f_H$.
  \end{enumerate}
  \begin{proof}
    $H$-localisations being initial is proven analoguesly to
    \enquote{$\Leftarrow$} in \ref{thm:charhlocal}, with the \Hiso{}
    as $g$. This then shows functoriality, and exactness follows using
    the five lemma. 
  \end{proof}
\end{Cor}

\section{Special Case: $\Ps$-Localisation}
Now let $H\coloneqq \LP\coloneqq \pistar{-}\timesZ$ for
$\Ps\subset\PS$, which is a homology functor represented by the
$\Z[\Ps]$-Moore-spectrum \cite[p.\,111, Exc. (a)]{margolis}.
This will be our means to transport localisation from homotopy groups
to spectra.
Write $X_{\Ps}$ for the $\LP$-localisation of $X$.

\begin{Prop}
  $f$ is the $\LP$-localisation \ifftext{}
  $\pistar f\colon \pistar X\to \pistar Y$ is a
  $\Ps$-localisation map.
  Or equivalently:
  $Y$ is $\LP$-local \ifftext{}
  $\pistar Y$ is $\Ps$-local, \idest a $\Z[\Ps]$-module.
  \begin{proof}
    \begin{description}
    \item[\enquote{$\Rightarrow$}] Let $Y$ be $\LP$-local.
      Being a $\Z[\Ps]$-module is equivalent to $(-\cdot q)$ (the
      multiplication by $q$) being an isomorphism for $q\in\PS\setminus\Ps$.
      So for $q\in\PS\setminus\Ps$ consider the $q$-times pinching map
      $\times q=\sum_{i=1}^q \id_Y$, which is $(-\cdot q)$ on homotopy,
      and thus an isomorphism on $\LP(Y)$.
      By exactness of $\LP$, $\Cone{\times q}$ is $\LP$-acyclic,
      especially the $\LP$-localisation of it is 0.
      By exactness of the $\LP$-localisation, with $Y_{\Ps}=Y$, and
      $(\times q)_{\Ps}=\times q$, we see that $\times q$ is an
      isomorphism. Thus also $(-\cdot q)\colon \pistar Y \to \pistar Y$.
    \item[\enquote{$\Leftarrow$}] Let $\pistar Y$ be a $\Z[\Ps]$-module and
      $W\in\Spectra$ $\LP$-acyclic.
      To see that any morphism $f\in [W,Y]$ is zero, one first has to
      conclude that the $\LP$-localisation $l_Y$ of $Y$ is an
      isomorphism. But this follows with Whitehead's Theorem for
      spectra from the isomorphism in the localisation diagram
      \begin{center}
        \begin{tikzcd}
          \pistar Y \ar[r]
          \ar[d, "\cong", "\text{ by ass.}" swap]
          \arrow[dr, phantom, "\circlearrowleft" description]
          & \pistar Y_{\Ps}
          \ar[d, "\cong", "\text{ by "$\Rightarrow$"}" swap] \\
          \pistar{Y}\timesZ \ar[r, "\cong", "\LP(l_Y)" swap]
          & \pistar{Y_{\Ps}}\timesZ
        \end{tikzcd}
      \end{center}
      Now $f$ will always factor over its $\LP$-localisation $f_{\Ps}$
      which is 0, as $W_{\Ps}$ is 0.
    \end{description}
  \end{proof}
\end{Prop}

\begin{Rem}
  Having adapted the notion of prime localisation to spectra one finds
  that we have very similar properties for spectra concerning the
  detection of torsion. We can now define torsion spectra as those whose
  $\LP[(0)]$-localisation (\idest rational homotopy groups) vanishes.
  Having only torsion for primes in a certain $\Ps\subset\PS$ (the
  $\Ps$-primary spectra) can be characterised as being
  $\LP[\Ps^c]$-acyclic.

  Analoguesly to purely torsion abelian groups, torsion spectra have the
  property that localisation at any set of primes $\Ps\subset\PS$ will
  be an epimorphism (for groups this means: elements are either already
  invertible or torsion; there is no free part).
  \begin{proof}
    To check this for a torsion spectrum $T$ with $\LP$-localisation
    $l_T$ just verify that
    \begin{enumerate}[nosep]
    \item the $\LP$-localisation $T_{\Ps}$ is
      $\LP[\Ps^c]$-acyclic, and
    \item the cone $\Cone {l_T}$ is $\LP[\Ps^c]$-local
      (it is isomorphic to $T_{\Ps^c}$ by exactness of
      localisation).
    \end{enumerate}
    Then $\conem {l_T} \in [T_{\Ps}, \Cone {l_T}]=0$.
  \end{proof}
\end{Rem}

\section{$\Ps$-Completion}

\begin{Rev}[$\Ps$-Completion of Groups]
  \begin{itemize}
  \item The \newdef{$\Ps$-completion} $\compl{G}$ of a $\Z$-module $G$
    is defined as the projective limit $\projlim_{G\to H} H$ over all
    morphisms from $G$ to finite $\Z$-modules $H$ with order prime to
    all $q\in\PS\setminus\Ps$.
    We get a canonical $\Ps$-completion morphism $G\to\compl{G}$.
  \item $\Ps$-completion factors over $\Ps$-localisation, \idest there
    always is a canonical morphism $G_\Ps\to\compl{G}$.
  \item If one restricts to graded $\Z$-modules, $\Ps$-completion
    preserves the grading.
  \item We call a \newdef{$\Ps$-sequence} a sequence $(n_i)_{i\in\N}$ of
    integers that satisfies for all $i\in\N$
    \begin{itemize}
    \item $n_i|n_{i+1}$
    \item $n_i$ is prime to all $q$ in $\PS\setminus\Ps$, and
    \item for any $n\in\Z$ with above property there is $i\in\N$ with $n|n_i$.
    \end{itemize}
    For a finitely generated $\Z$-module $G$ there are isomorphisms
    \begin{gather*}
      \compl{G}
      \cong \varprojlim_i G/n_iG
      \cong G\otimes\compl{\Z}
    \end{gather*}
    which offers a simpler construction of $\compl{G}$.
  \end{itemize}
\end{Rev}
\begin{Rev}[Limits]
  A useful notion of limits of spectra is the following:
  The \newdef{weak limit} of a sequence
  $(X_i, f_i\colon X_i\to X_{i+1})_{i\in\N}$ of spectra is defined
  by the cofibre sequence
  \begin{gather*}
    \wlim_{i} X_i
    \longto \prod_i X_i
    \xrightarrow{\prod_i (id-f_i)\circ \pi_i} \prod_i X_i
    \;.
  \end{gather*}
  If $\pistar[r]{X_i}$ is finite for any degree $r\in\N$, we have
  $\projlim_i \pistar[r]{X_i} \cong \pistar[r]{\wlim_i X_i}$.
\end{Rev}
\begin{Def}
  A morphism $f\colon X\to Y$ of spectra is called a
  \newdef{$\Ps$-completion} of $X$ if in all degrees $r\in\N$
  $\pistar[r]f\colon\pistar[r] X\to\pistar[r] Y$ is a
  $\Ps$-completion of $\Z$-modules.
\end{Def}
\begin{Lem}
  $\Ps$-completions are unique up to equivalence, and they exist for
  spectra whose $\Ps$-localisation admits homotopy groups finitely
  generated over $\Z[\Ps]$.
  \begin{proof}
    For further details and uniqueness see \cite[Thm. 9.1]{margolis}.

    As $\Ps$-completion factors over $\Ps$-localisation, without loss
    of generality $X$ is $\Ps$-local, and thus has finitely
    generated $\Z[\Ps]$-modules as homotopy groups.
    For existence then consider the following construction which
    immitates the simpler construction of completion for groups:
    \begin{Def}
      Let $n\in\N$ and $(n_i)_{i\in\N}$ a $\Ps$-sequence.
      \begin{itemize}
      \item Similarly to groups, where
        $G/n\coloneqq \operatorname{coim}(G\xrightarrow{\cdot n} G)$,
        define
        $\Sph/n\coloneqq\Cone{\Sph\xrightarrow{\times n}\Sph}$.
      \item Define $X/n\coloneqq X\smashed S/n
        \cong \Cone{X\xrightarrow{\times n} X}$.
      \item Define 
        $\compl{X}\coloneqq\wlim_{i} X/n_i$.
        This admits canonical morphisms $X/n\xrightarrow{\text{pr}} X/m$ for $m|n$.
      \end{itemize}
    \end{Def}
    \begin{Lem}
      $\compl{X}$ is a completion of $X$.
      \begin{proof}
        The long exact sequence of homotopy corresponding to the
        cofibre sequence $X\xrightarrow{\times n_i}X\to X/n_i$
        yields for any degree $r\in\N$ the following short exact
        sequences, using
        $\operatorname{coim}(\cdot n_{i-1}) = \pistar[r]{X}/n_{i-1}$
        and
        $\pistar[r-1]{X}[n_i] = \operatorname{ker}(\cdot n_i)$:
        \begin{center}
          \begin{tikzcd}
            \vdots \arrow[d, equals]
            &\vdots \arrow[d]
            &\vdots \arrow[d]
            &\vdots \arrow[d]
            &\vdots \arrow[d, equals]
            \\
            0\arrow[r]\arrow[d, equals]
            &\pistar[r]{X}/n_i \arrow[r]
            \arrow[d, "\mod n_i" swap]
            \arrow[dr, phantom, "\circlearrowleft" description]
            & \pistar[r]{X/n_i} \arrow[r]
            \arrow[d, "{\pistar[r]{\text{pr}}}"]
            \arrow[dr, phantom, "\circlearrowleft" description]
            & \pistar[r-1]{X}[n_i] \arrow[r]
            \arrow[d, "{\cdot \frac{n_i}{n_{i-1}}}"]
            & 0 \arrow[d, equals]
            \\
            0\arrow[r]\arrow[d, equals]
            &\pistar[r]{X}/n_{i-1} \arrow[r]\arrow[d]
            & \pistar[r]{X/n_{i-1}} \arrow[r]\arrow[d]
            & \pistar[r-1]{X}[n_{i-1}] \arrow[r]\arrow[d]
            & 0 \arrow[d, equals]
            \\
            \vdots
            &\vdots
            &\vdots
            &\vdots
            &\vdots
          \end{tikzcd}
        \end{center}
        By left-exactness of the limit, we get an exact sequence
        \begin{center}
          \begin{tikzcd}
            0\ar[r]
            & \varprojlim_i \pistar[r]{X}/n_i \ar[r]
            \ar[d, equals, "{\text{Def.}}"]
            & \varprojlim_i \pistar[r]{X/n_i} \ar[r]
            \ar[d, "\cong" swap, "{\pistar[r-1]{X}\text{ finite}}"]
            & \varprojlim_i \pistar[r-1]{X}[n_i] 
            \ar[d, equals, "{\pistar[r-1]{X}\text{ finite}}"]
            \\
            & \compl{\pistar[r]{X}}
            & \pistar[r]{\compl{X}}
            & 0
          \end{tikzcd}\qedhere
        \end{center}
      \end{proof}
    \end{Lem}
  \end{proof}
\end{Lem}

Until now we have ported the easy construction of $\Ps$-completion of
finitely generated groups to a limited amount of spectra. However,
getting a notion of completion on a general spectrum, on requires to
port the more general definition.
\begin{DefLem}
  \begin{itemize}
  \item Let $\Spectra_f$ be the full subcategory of all spectra with
    finite homotopy groups, and $\Spectra_f'$ a small, equivalent
    subcategory. 
    Let $\A$ be the full subcategory of $\Spectra_f'$ of spectra with
    $\Ps$-torsion homotopy groups.
  \item Let $\Gamma_\A(X)$ be the category of morphisms $X\to W$
    with $W\in\A$.
  \item The mapping $Y\mapsto\lim_{\Gamma_\A(X)}[Y, W]$ is a
    cohomology functor \cite[Prop. 9.6]{margolis}. Thus, it is
    represented by a spectrum $\compl[\A]X$ which we call the
    \newdef{pro $\A$-completion}.
    Note, that by definition
    $\pistar{\compl[\A]X} \cong \lim_{(X\to W)\in\Gamma_\A(X)}\pistar W$.
  \end{itemize}
\end{DefLem}
\begin{Thm}
  For a $\Ps$-local spectrum $X$ whose homotopy groups are finitely
  generated $\Z[\Ps]$-modules holds $\compl[\A]{X}\cong\compl{X}$,
  \idest the pro $\A$-completion actually generalises $\Ps$-completion.
  \begin{proof}
    Note, that with $X/n\in\A$ for any $n\in\Z$ we have canonical
    maps $\compl[\A]X \to X/n_i$, and thus a canonical map
    $\compl[\A]X\xrightarrow{\alpha}\compl X$. We will show that this
    is an equivalence by showing that it is an isomorphism on homotopy
    groups:
    \begin{align*}
      \lim_{\mathclap{(f\colon X\to W\in\Gamma_\A(X)}}\pistar W \cong \pistar{\compl[\A]X}
      &\longto
        \pistar{\compl X} \cong \lim_i\pistar{X/n_i}
      \\
      (x_f)_{(f\colon X\to W)\in\Gamma_\A(X)}
      &\longmapsto
        (x_{\text{pr}_i})_{i\in\N}
    \end{align*}
    Beforehands, note the following factorisation for any morphism
    $f\colon X\to W$ of spectra and any $n\in\N$:
    \begin{center}
      \begin{tikzcd}
        & W \ar[rd, "\simeq"]
        \\
        X \ar[ru, "f"] \ar[rd, "\text{pr}"]
        \ar[rr, phantom, "\circlearrowleft" description]
        && W/n 
        \\
        & X/n \ar[ru]
      \end{tikzcd}
    \end{center}
    \begin{description}
    \item[$\alpha$ injective] Consider an element
      $x=(x_f\colon \Sph^n \to W)_f \in\pistar{\compl[\A]X}$.
      If this is non-trivial, there must be $W$ and $f\colon X\to W$
      \suchthat $x_f$ is non-trivial. Using $W\simeq W/n_i$ for
      $i$ resp. $n_i$ large enough (look at homotopy groups) the
      following commutative diagram yields that, for such $i$,
      $x_{\text{pr}_i}$ is non-trivial --- and thus also $\alpha(x)$:
      \begin{center}
        \begin{tikzcd}
          && W \ar[rd, "\simeq"]
          \\
          \Sph^n
          \ar[urr, "x_f", bend left]
          \ar[drr, "x_{\text{pr}_i}", bend right]
          \ar[r, phantom, "\circlearrowleft" description]
          & X \ar[ru, "f"] \ar[rd, "\text{pr}_i"]
          \ar[rr, phantom, "\circlearrowleft" description]
          && W/n_i 
          \\
          &~& X/n \ar[ru]
        \end{tikzcd}
      \end{center}
      The left part commutes by definition, the right part comes from
      the canonical factorisation.
    \item[$\alpha$ surjective] Let
      $\bar x=(x_{\text{pr}_i})_i\in\pistar{\compl X}$.
      For any $W$ and $f\colon X\to W$ in $\Gamma_\A(X)$ choose $i\gg0$,
      $n_f\coloneqq n_i$, \suchthat $W\cong W/n_f$.
      Define $f/n_f\colon X\to W\cong W/n_f$
      and set $x\coloneqq (x_f\coloneqq (f/n_f)^*x_{\text{pr}_i})_f$.
      Then $\alpha(x)=\bar x$.
      \qedhere
    \end{description}
  \end{proof}
\end{Thm}

\nocite{*}
\printbibliography
\end{document}

